\PassOptionsToPackage{table}{xcolor}

\documentclass{beamer}
 
\usepackage[utf8]{inputenc}
\usetheme{Madrid}
\usepackage{tcolorbox}
\usepackage{tabularx}
\definecolor{RowColorOdd}{rgb}{0.914,0.914,0.953}
\definecolor{RowColorEven}{rgb}{1,1,1}
\usepackage[percent]{overpic}
\usepackage{tikz}
 
%Information to be included in the title page:
\title[Housework Research] %optional
{Decomposing Housework in the US and Canada}
 
\subtitle{Gendered Division of Housework Tasks}
 
\author[Kolpashnikova, Kamila] % (optional, for multiple authors)
{Kamila Kolpashnikova\inst{1}}
 
\institute[NTPU] % (optional)
{
  \inst{1}%
  Visiting Scholar\\
  Department of Sociology\\
  National Taipei University  
}
 
\date[Seminar 2018] % (optional)
{NTPU Seminar, 30 May 2018}
 
\logo{\includegraphics[width=3cm]{ntpu-logo01.jpg}}


 
 
 
\begin{document}
 
\frame{\titlepage}
 
\begin{frame}
\frametitle{Structure of the Talk (2018/05/30)}
  \begin{enumerate}
  	\item  Decomposition of housework and application to the data in the US and Canada 
  	\begin{itemize}
  	\item Time use research
  	\item Importance of the private sphere for gender equality
  	\item Cultural differences and housework research
  	\end{itemize}
  \item Other Countries
    \begin{itemize}
  	\item Housework in ISSP Countries, including
  	\item Gender in Taiwan
  	\end{itemize}
  \item  What are next steps?
  	\begin{itemize}
  	\item Future research
  	\item Collaborative initiatives
  	\end{itemize}
  \end{enumerate}
\end{frame}

\begin{frame}
\frametitle{Bargaining? Autonomous? Doing Gender?}
  \begin{enumerate}
  	\item  Economic approach. [More resources = less housework] 
  	\begin{itemize}
  	\item \textbf{Relative resources} a.k.a. bargaining (Brines 1994; Greenstein 2000)
  	\item Time constraint, or \textbf{time availability} (Artis and Pavalko 2003; Coverman 1985; Presser 1994; Silver and Goldscheider 2013)
  	\item \textbf{Autonomy} (Gupta 2007; Killewald and Gough, 2010)
Gender works indirectly through resources (Bittman, England, and Sayer 2003).
  	\end{itemize}
  \item Gender-centred approach. [gender performance, women do housework as a part of performing their gender]
    \begin{itemize}
  	\item Household is a ‘site for doing gender’ (West and Zimmerman 1987) or a ‘gender factory’ (Berk 1985)
  	\item sex/gender attitudes → gender socialization → gender performance
  	\item  \textbf{“doing gender”} (West and Zimmerman 1987; Butler 1990)
  	\item \textbf{“gender ideology”} (Davis and Greenstein, 2009)
  	\end{itemize}
  \end{enumerate}
\end{frame}


\begin{frame}
\frametitle{Contributions}
  	\begin{itemize}
  	\item Theoretically applying the framework as they pertain to the GAP not the differences among women and among men.
  	\item New analysis of participation in housework by cultural groups
  	\item Novel method in the analysis of the gender gap in housework – decomposition. \\~\\
  	
  	\hspace*{20pt} To this day there is only a handful of studies (Pepin, Sayer, and Casper, 2018; Kim and Chin, 2016) that systematically address the analysis of factors explaining the gap in the division of housework tasks itself

  	\end{itemize}
\end{frame}

\begin{frame}
\frametitle{Background}
  	\begin{itemize}
  	\item Convergence is happening albeit slower than expected (Kan et al., 2011; Marshall, 2011; Sullivan, Gershuny, and Robinson, 2018; Altinas and Sullivan, 2016)
  	\item NB: Previous studies mostly report aggregate housework. 

  	\end{itemize}
\end{frame}

\begin{frame}
\frametitle{Question \#1}

The time women and men spend on aggregate housework might be converging but no study has yet analyzed the gap comparing men vs women. \\~\\

\begin{tcolorbox}[colback=green!5,colframe=green!40!black,title=Question 1] 
\#1. Are there differences by housework tasks?
\end{tcolorbox}

\end{frame}

\begin{frame}
\frametitle{Question \#2}

Moreover, the assumption is that participation in each task would have the same explanation.
 \\~\\

\begin{tcolorbox}[colback=green!5,colframe=green!40!black,title=Question 2] 
\#2. Can theoretical frameworks explain the gender gap in all different housework tasks?  

\end{tcolorbox}

\end{frame}

\begin{frame}
\frametitle{Question \#3}

There should be some differences by different cultural groups.
 \\~\\

\begin{tcolorbox}[colback=green!5,colframe=green!40!black,title=Question 2] 
\#3. Can theoretical frameworks explain the gender gap in all ethnic groups equally?

\end{tcolorbox}

\end{frame}

\begin{frame}
\frametitle{By gendered character (Doing Gender)}

\begin{tcolorbox}[colback=blue!5,colframe=blue!40!black,title=Hypothesis 1] 
H1. The gender gap is expected to be lower in gender-neutral tasks than in traditionally gendered.
\end{tcolorbox}


The biggest gap is expected in the gendered and the least pleasant task.
\end{frame}

\begin{frame}
\frametitle{The Gender Gap: What we would expect by different tasks?}

\begin{tcolorbox}[colback=blue!5,colframe=blue!40!black,title=Hypothesis 2.1] 
H2.1. The gender gap is more likely to be explained by the availability of resources, relative or absolute, including time, for the tasks that are more gender neutral such as shopping compared to tasks that are still considered more typically feminine – such as routine indoor housework tasks, cooking and cleaning.
\end{tcolorbox}

\end{frame}

\begin{frame}
\frametitle{Power is expected to be relevant for more gendered types of housework:}

\begin{tcolorbox}[colback=blue!5,colframe=blue!40!black,title=Hypothesis 2.2] 
H2.2. Relative resources are expected to account for more of the gender gap than absolute resources or time availability in all gendered tasks such as routine indoor housework tasks, cooking and cleaning, compared to gender-neutral tasks such as shopping.
\end{tcolorbox}

\end{frame}

\begin{frame}
\frametitle{Cultural differences}

\begin{tcolorbox}[colback=blue!5,colframe=blue!40!black,title=Hypothesis 3.1] 
H3.1.: The gender gap is more likely to be explained by the availability of resources, relative or absolute, including time, among the dominant groups than among minority groups.
\end{tcolorbox}

\begin{tcolorbox}[colback=blue!5,colframe=blue!40!black,title=Hypothesis 3.2] 
H3.2.: Relative resources are expected to account for more of the gender gap than absolute resources or time availability among minority groups rather than the dominant groups.
\end{tcolorbox}

\end{frame}

\begin{frame}
\frametitle{Data and Methods}

Question \#1) Analysis of the absolute difference/proportion

Question \#2) Blinder-Oaxaca decomposition method 

Question \#3) Both 1/2
\\~\\
Datasets used: 
  	\begin{itemize}
  	\item Canadian General Social Survey, 1986-2010
  	\item American Time Use Survey, 2003-2015
   	\end{itemize}
\end{frame}

\begin{frame}
\frametitle{Sample}

  	\begin{itemize}
  	\item married individuals or those who are in common law relationships. 
  	\item the one-person households were dropped out of the analysis 
  	\item person-days
  	\item Personal weights were re-coded based on the original survey weights and scaled to the original sample size before the sub-setting. 
   	\end{itemize}
\end{frame}

\begin{frame}
\frametitle{Variables}
  	\begin{itemize}
  	\item Dependent: Time spent on cooking, cleaning, shopping, and maintenance tasks
  	\item Independent (Relative resources): Income Transfer
  	\item Independent (Time Availability): time spent on paid work, children, children under 5
  	\item Independent (Autonomy): Personal Income
	\item Independent (Doing gender): by character of the tasks, gender expectations traditionalism in cultural groups, Education and Age
	\item Other controls: the rest of usual suspects and province/state level variables
   	\end{itemize}
\end{frame}

\begin{frame}
\frametitle{H1.1. Gender gap narrower in gender-neutral tasks?}

\tiny
  \begin{block}{\centering\tiny\begin{tabularx}{\dimexpr\textwidth-2\tabcolsep}{@{}X@{}p{0.30\textwidth}@{}X@{}X@{}X@{}X@{}X@{}}\textcolor{white}{Variables} & 
\textcolor{white}{Description} & 
\textcolor{white}{Mean 
(Women)}& 
\textcolor{white}{Bootstrap SE}& 
\textcolor{white}{Mean 
(Men)}& 
\textcolor{white}{Bootstrap SE}& 
\textcolor{white}{Diff. 
in Means}
\end{tabularx}}%
  \centering
    \rowcolors{1}{RowColorOdd}{RowColorEven}%
    \begin{tabularx}{\dimexpr\textwidth-2\tabcolsep}{@{}X@{}p{0.30\textwidth}@{}X@{}X@{}X@{}X@{}X@{}}%
      Domestic Tasks
 & Domestic tasks combined (min.)
 & 206.02
 & (0.348)
 & 112.79
 & (0.280)
 & 93.23{\textsuperscript{***}}
 \\%
       Cooking Tasks
 & Cooking and washing up (min.)
 & 72.320
 & (0.146)
 & 26.830
 & 26.830
 & 45.49{\textsuperscript{***}}
 \\%
      Cleaning Tasks & 
      Cleaning, laundry, and mending (min.)
 & 70.373
 & (0.195)
 & 20.384
 & (0.112)
 & 49.989{\textsuperscript{***}}
 \\%
       Maintenance Tasks
 & Maintenance and repair (min.)
 & 6.045
 & (0.074)
 & 22.214
 & (0.148)
 & -16.169{\textsuperscript{***}}
 \\%
      Shopping Tasks & 
      Shopping and services (min.)
 & 57.281
 & (0.211)
 & 43.362
 & (0.178)
 & 13.9190{\textsuperscript{***}}
 
    \end{tabularx}%

  \end{block}%
  \normalsize
  This is approximately 567 hours more in a year, a bit $<$ a month 
 


\end{frame}

\begin{frame}
\frametitle{H1.1. (US)}

\tiny
  \begin{block}{\centering\tiny\begin{tabularx}{\dimexpr\textwidth-2\tabcolsep}{@{}X@{}p{0.30\textwidth}@{}X@{}X@{}X@{}X@{}X@{}}\textcolor{white}{Variables} & 
\textcolor{white}{Description} & 
\textcolor{white}{Mean 
(Women)}& 
\textcolor{white}{Robust SE}& 
\textcolor{white}{Mean 
(Men)}& 
\textcolor{white}{Robust SE}& 
\textcolor{white}{Diff. 
in Means}
\end{tabularx}}%
  \centering
    \rowcolors{1}{RowColorOdd}{RowColorEven}%
    \begin{tabularx}{\dimexpr\textwidth-2\tabcolsep}{@{}X@{}p{0.30\textwidth}@{}X@{}X@{}X@{}X@{}X@{}}%
      Domestic Tasks
 & Domestic tasks combined (min.)
 & 187.539
 & (0.909)
 & 96.632
 & (0.773)
 & 90.907{\textsuperscript{***}}
 \\%
       Cooking Tasks
 & Cooking and washing up (min.)
 & 58.627
 & (0.386)
 & 19.775
 & (0.241)
 & 38.852{\textsuperscript{***}}
 \\%
      Cleaning Tasks & 
      Cleaning, laundry, and mending (min.)
 & 69.295
 & (0.605)
 & 24.031
 & (0.412)
 & 45.264{\textsuperscript{***}}
 \\%
       Maintenance Tasks
 & Maintenance and repair (min.)
 & 2.272
 & (0.123)
 & 12.842
 & (0.393)
 & -10.57{\textsuperscript{***}}
 \\%
      Shopping Tasks & 
      Shopping and services (min.)
 & 57.345
 & (0.509)
 & 39.985
 & (0.442)
 & 17.360{\textsuperscript{***}}
 
    \end{tabularx}%

  \end{block}%
  \normalsize
  This is approximately 553 hours more in a year, a bit $<$ a month

\end{frame}
 
\begin{frame}
\frametitle{Cooking}
  \includegraphics[width=\textwidth]{figure5.jpeg}
\end{frame}

\begin{frame}
\frametitle{Cleaning}
  \includegraphics[width=\textwidth]{figure6.jpeg}
\end{frame}

\begin{frame}
\frametitle{Shopping}
  \includegraphics[width=\textwidth]{figure7.jpeg}
\end{frame}

\begin{frame}
\frametitle{Maintenance}
  \includegraphics[width=\textwidth]{figure8.jpeg}
\end{frame}

\end{document}

