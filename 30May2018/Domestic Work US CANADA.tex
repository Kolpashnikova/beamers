\PassOptionsToPackage{table}{xcolor}

\documentclass{beamer}
 
\usepackage[utf8]{inputenc}
\usetheme{Madrid}
\usepackage{tcolorbox}
\usepackage{tabularx}
\definecolor{RowColorOdd}{rgb}{0.914,0.914,0.953}
\definecolor{RowColorEven}{rgb}{1,1,1}
\usepackage[percent]{overpic}
\usepackage{tikz}
\usepackage{multirow}
\usepackage[export]{adjustbox}



%Information to be included in the title page:
\title[Housework Research] %optional
{Decomposing Housework in the US and Canada}
 
\subtitle{Gendered Division of Housework Tasks}
 
\author[Kolpashnikova, Kamila] % (optional, for multiple authors)
{Kamila Kolpashnikova\inst{1}, Man Yee Kan \inst{2}, \& Jooyeoun Suh \inst{3}}
 
\institute[NTPU] % (optional)
{
  \inst{1}%
  Visiting Scholar\\
  Department of Sociology\\
  National Taipei University  \\~\\
  \inst{2}%
  Department of Sociology\\
  University of Oxford  \\~\\
  \inst{3}%
  Institute for Women's Policy Research 
}
 
\date[Seminar 2018] % (optional)
{NTPU Seminar, 30 May 2018}
 
 
 
 
\begin{document}
 
\frame{\titlepage}
 
\begin{frame}
\frametitle{Structure of the Talk (2018/05/30)}
  \begin{enumerate}
  	\item  Decomposition of housework and application to the data in the US and Canada 
  	\begin{itemize}
  	\item Time use research
  	\item Importance of the private sphere for gender equality
  	\item Cultural differences and housework research
  	\end{itemize}
  \item Other Countries
    \begin{itemize}
  	\item Housework in ISSP Countries, including
  	\item Housework in Japan
  	\item Tempograms
  	\end{itemize}
  \item  What are next steps?
  	\begin{itemize}
  	\item Future research
  	\item Collaborative initiatives
  	\end{itemize}
  \end{enumerate}
\end{frame}

\begin{frame}
\frametitle{Bargaining? Autonomous? Doing Gender?}
  \begin{enumerate}
  	\item  Economic approach. [More resources = less housework] 
  	\begin{itemize}
  	\item \textbf{Relative resources} a.k.a. bargaining (Brines 1994; Greenstein 2000)
  	\item Time constraint, or \textbf{time availability} (Artis and Pavalko 2003; Coverman 1985; Presser 1994; Silver and Goldscheider 2013)
  	\item \textbf{Autonomy} (Gupta 2007; Killewald and Gough, 2010)
Gender works indirectly through resources (Bittman, England, and Sayer 2003).
  	\end{itemize}
  \item Gender-centred approach. [gender performance, women do housework as a part of performing their gender]
    \begin{itemize}
  	\item Household is a ‘site for doing gender’ (West and Zimmerman 1987) or a ‘gender factory’ (Berk 1985)
  	\item sex/gender attitudes → gender socialization → gender performance
  	\item  \textbf{“doing gender”} (West and Zimmerman 1987; Butler 1990)
  	\item \textbf{“gender ideology”} (Davis and Greenstein, 2009)
  	\end{itemize}
  \end{enumerate}
\end{frame}


\begin{frame}
\frametitle{Contributions}
  	\begin{itemize}
  	\item Theoretically applying the framework as they pertain to the GAP not the differences among women and among men.
  	\item New analysis of participation in housework by cultural groups
  	\item Novel method in the analysis of the gender gap in housework – decomposition. \\~\\
  	
  	\hspace*{20pt} To this day there is only a handful of studies (Pepin, Sayer, and Casper, 2018; Kim and Chin, 2016) that systematically address the analysis of factors explaining the gap in the division of housework tasks itself

  	\end{itemize}
\end{frame}

\begin{frame}
\frametitle{Background}
  	\begin{itemize}
  	\item Convergence is happening albeit slower than expected (Kan et al., 2011; Marshall, 2011; Sullivan, Gershuny, and Robinson, 2018; Altinas and Sullivan, 2016)
  	\item NB: Previous studies mostly report aggregate housework. 

  	\end{itemize}
\end{frame}

\begin{frame}
\frametitle{Question \#1}

The time women and men spend on aggregate housework might be converging but no study has yet analyzed the gap comparing men vs women. \\~\\

\begin{tcolorbox}[colback=green!5,colframe=green!40!black,title=Question 1] 
\#1. Are there differences by housework tasks?
\end{tcolorbox}

\end{frame}

\begin{frame}
\frametitle{Question \#2}

Moreover, the assumption is that participation in each task would have the same explanation.
 \\~\\

\begin{tcolorbox}[colback=green!5,colframe=green!40!black,title=Question 2] 
\#2. Can theoretical frameworks explain the gender gap in all different housework tasks?  

\end{tcolorbox}

\end{frame}

\begin{frame}
\frametitle{Question \#3}

There should be some differences by different cultural groups.
 \\~\\

\begin{tcolorbox}[colback=green!5,colframe=green!40!black,title=Question 2] 
\#3. Can theoretical frameworks explain the gender gap in all ethnic groups equally?

\end{tcolorbox}

\end{frame}

\begin{frame}
\frametitle{By gendered character (Doing Gender)}

\begin{tcolorbox}[colback=blue!5,colframe=blue!40!black,title=Hypothesis 1] 
H1. The gender gap is expected to be lower in gender-neutral tasks than in traditionally gendered.
\end{tcolorbox}


The biggest gap is expected in the gendered and the least pleasant task.
\end{frame}

\begin{frame}
\frametitle{The Gender Gap: What we would expect by different tasks?}

\begin{tcolorbox}[colback=blue!5,colframe=blue!40!black,title=Hypothesis 2.1] 
H2.1. The gender gap is more likely to be explained by the availability of resources, relative or absolute, including time, for the tasks that are more gender neutral such as shopping compared to tasks that are still considered more typically feminine – such as routine indoor housework tasks, cooking and cleaning.
\end{tcolorbox}

\end{frame}

\begin{frame}
\frametitle{Power is expected to be relevant for more gendered types of housework:}

\begin{tcolorbox}[colback=blue!5,colframe=blue!40!black,title=Hypothesis 2.2] 
H2.2. Relative resources are expected to account for more of the gender gap than absolute resources or time availability in all gendered tasks such as routine indoor housework tasks, cooking and cleaning, compared to gender-neutral tasks such as shopping.
\end{tcolorbox}

\end{frame}

\begin{frame}
\frametitle{Cultural differences}

\begin{tcolorbox}[colback=blue!5,colframe=blue!40!black,title=Hypothesis 3.1] 
H3.1.: The gender gap is more likely to be explained by the availability of resources, relative or absolute, including time, among the dominant groups than among minority groups.
\end{tcolorbox}

\begin{tcolorbox}[colback=blue!5,colframe=blue!40!black,title=Hypothesis 3.2] 
H3.2.: Relative resources are expected to account for more of the gender gap than absolute resources or time availability among minority groups rather than the dominant groups.
\end{tcolorbox}

\end{frame}

\begin{frame}
\frametitle{Data and Methods}

Question \#1) Analysis of the absolute difference/proportion

Question \#2) Blinder-Oaxaca decomposition method 

Question \#3) Both 1/2
\\~\\
Datasets used: 
  	\begin{itemize}
  	\item Canadian General Social Survey, 1986-2010
  	\item American Time Use Survey, 2003-2015
   	\end{itemize}
\end{frame}

\begin{frame}
\frametitle{Sample}

  	\begin{itemize}
  	\item married individuals or those who are in common law relationships. 
  	\item the one-person households were dropped out of the analysis 
  	\item person-days
  	\item Personal weights were re-coded based on the original survey weights and scaled to the original sample size before the sub-setting. 
   	\end{itemize}
\end{frame}

\begin{frame}
\frametitle{Variables}
  	\begin{itemize}
  	\item Dependent: Time spent on cooking, cleaning, shopping, and maintenance tasks
  	\item Independent (Relative resources): Income Transfer
  	\item Independent (Time Availability): time spent on paid work, children, children under 5
  	\item Independent (Autonomy): Personal Income
	\item Independent (Doing gender): by character of the tasks, gender expectations traditionalism in cultural groups, Education and Age
	\item Other controls: the rest of usual suspects and province/state level variables
   	\end{itemize}
\end{frame}

\begin{frame}
\frametitle{H1.1. Gender gap narrower in gender-neutral tasks?}

\tiny
  \begin{block}{\centering\tiny\begin{tabularx}{\dimexpr\textwidth-2\tabcolsep}{@{}X@{}p{0.30\textwidth}@{}X@{}X@{}X@{}X@{}X@{}}\textcolor{white}{Variables} & 
\textcolor{white}{Description} & 
\textcolor{white}{Mean 
(Women)}& 
\textcolor{white}{Bootstrap SE}& 
\textcolor{white}{Mean 
(Men)}& 
\textcolor{white}{Bootstrap SE}& 
\textcolor{white}{Diff. 
in Means}
\end{tabularx}}%
  \centering
    \rowcolors{1}{RowColorOdd}{RowColorEven}%
    \begin{tabularx}{\dimexpr\textwidth-2\tabcolsep}{@{}X@{}p{0.30\textwidth}@{}X@{}X@{}X@{}X@{}X@{}}%
      Domestic Tasks
 & Domestic tasks combined (min.)
 & 206.02
 & (0.348)
 & 112.79
 & (0.280)
 & 93.23{\textsuperscript{***}}
 \\%
       Cooking Tasks
 & Cooking and washing up (min.)
 & 72.320
 & (0.146)
 & 26.830
 & (0.373)
 & 45.49{\textsuperscript{***}}
 \\%
      Cleaning Tasks & 
      Cleaning, laundry, and mending (min.)
 & 70.373
 & (0.195)
 & 20.384
 & (0.112)
 & 49.989{\textsuperscript{***}}
 \\%
       Maintenance Tasks
 & Maintenance and repair (min.)
 & 6.045
 & (0.074)
 & 22.214
 & (0.148)
 & -16.169{\textsuperscript{***}}
 \\%
      Shopping Tasks & 
      Shopping and services (min.)
 & 57.281
 & (0.211)
 & 43.362
 & (0.178)
 & 13.9190{\textsuperscript{***}}
 
    \end{tabularx}%

  \end{block}%
  \normalsize
  This is approximately 567 hours more in a year, a bit $<$ a month 
 


\end{frame}

\begin{frame}
\frametitle{H1.1. (US)}

\tiny
  \begin{block}{\centering\tiny\begin{tabularx}{\dimexpr\textwidth-2\tabcolsep}{@{}X@{}p{0.30\textwidth}@{}X@{}X@{}X@{}X@{}X@{}}\textcolor{white}{Variables} & 
\textcolor{white}{Description} & 
\textcolor{white}{Mean 
(Women)}& 
\textcolor{white}{Robust SE}& 
\textcolor{white}{Mean 
(Men)}& 
\textcolor{white}{Robust SE}& 
\textcolor{white}{Diff. 
in Means}
\end{tabularx}}%
  \centering
    \rowcolors{1}{RowColorOdd}{RowColorEven}%
    \begin{tabularx}{\dimexpr\textwidth-2\tabcolsep}{@{}X@{}p{0.30\textwidth}@{}X@{}X@{}X@{}X@{}X@{}}%
      Domestic Tasks
 & Domestic tasks combined (min.)
 & 187.539
 & (0.909)
 & 96.632
 & (0.773)
 & 90.907{\textsuperscript{***}}
 \\%
       Cooking Tasks
 & Cooking and washing up (min.)
 & 58.627
 & (0.386)
 & 19.775
 & (0.241)
 & 38.852{\textsuperscript{***}}
 \\%
      Cleaning Tasks & 
      Cleaning, laundry, and mending (min.)
 & 69.295
 & (0.605)
 & 24.031
 & (0.412)
 & 45.264{\textsuperscript{***}}
 \\%
       Maintenance Tasks
 & Maintenance and repair (min.)
 & 2.272
 & (0.123)
 & 12.842
 & (0.393)
 & -10.57{\textsuperscript{***}}
 \\%
      Shopping Tasks & 
      Shopping and services (min.)
 & 57.345
 & (0.509)
 & 39.985
 & (0.442)
 & 17.360{\textsuperscript{***}}
 
    \end{tabularx}%

  \end{block}%
  \normalsize
  This is approximately 553 hours more in a year, a bit $<$ a month

\end{frame}
 
\begin{frame}
\frametitle{Cooking}
  \includegraphics[width=\textwidth]{figure5.jpeg}
\end{frame}

\begin{frame}
\frametitle{Cleaning}
  \includegraphics[width=\textwidth]{figure6.jpeg}
\end{frame}

\begin{frame}
\frametitle{Shopping}
  \includegraphics[width=\textwidth]{figure7.jpeg}
\end{frame}

\begin{frame}
\frametitle{Maintenance}
  \includegraphics[width=\textwidth]{figure8.jpeg}
\end{frame}

\begin{frame}
\frametitle{H2.1. Gender gap explained more in gender-neutral tasks?}

\tiny
  \begin{block}{\centering\tiny\begin{tabularx}{\dimexpr\textwidth-2\tabcolsep}{@{}X@{}p{0.15\textwidth}@{}X@{}X@{}X@{}X@{}X@{}X@{}}\textcolor{white}{} & {} &
\textcolor{white}{All} &
\textcolor{white}{Anglo} & 
\textcolor{white}{French}& 
\textcolor{white}{Chinese}& 
\textcolor{white}{South Asian}& 
\textcolor{white}{Filipino}
\end{tabularx}}%
  \centering
    \begin{tabularx}{\dimexpr\textwidth-2\tabcolsep}{@{}X@{}p{0.15\textwidth}@{}X@{}X@{}X@{}X@{}X@{}X@{}}%
      {\multirow{2}{*}{Cook}}
 & explained
 & 14{\textsuperscript{***}}
 & 15{\textsuperscript{***}}
 & 9{\textsuperscript{***}}
 & 16{\textsuperscript{**}}
 & 19{\textsuperscript{***}}
 & 17{\textsuperscript{***}} \\
 & \cline{1-7}
 & unexplained
 & 86{\textsuperscript{***}}
 & 85{\textsuperscript{***}}
 & 91{\textsuperscript{***}}
 & 84{\textsuperscript{***}}
 & 81{\textsuperscript{***}}
 & 83{\textsuperscript{***}}
 \\~\\
\hline
     {\multirow{2}{*}{Clean}}
 & explained
 & 20{\textsuperscript{***}}
 & 21{\textsuperscript{***}}
 & 14{\textsuperscript{***}}
 & 11
 & 24{\textsuperscript{***}}
 & 28{\textsuperscript{***}} \\
 & \cline{1-7}
 & unexplained
 & 80{\textsuperscript{***}}
 & 79{\textsuperscript{***}}
 & 86{\textsuperscript{***}}
 & 89{\textsuperscript{***}}
 & 76{\textsuperscript{***}}
 & 72{\textsuperscript{***}}
 \\~\\
\hline
     {\multirow{2}{*}{Shop}}
 & explained
 & 38{\textsuperscript{***}}
 & 39{\textsuperscript{***}}
 & 24{\textsuperscript{***}}
 & -47
 & 80{\textsuperscript{***}}
 & 59{\textsuperscript{***}} \\
 & \cline{1-7}
 & unexplained
 & 62{\textsuperscript{***}}
 & 61{\textsuperscript{***}}
 & 76{\textsuperscript{***}}
 & 147
 & 20
 & 41{\textsuperscript{**}}
 \\~\\
\hline

{\multirow{2}{*}{Main}}
 & explained
 & -14{\textsuperscript{***}}
 & -16{\textsuperscript{***}}
 & -1
 & -3
 & 0
 & -20{\textsuperscript{***}} \\
 & \cline{1-7}
 & unexplained
 & 114{\textsuperscript{***}}
 & 116{\textsuperscript{***}}
 & 101{\textsuperscript{***}}
 & 103{\textsuperscript{**}}
 & 100{\textsuperscript{***}}
 & 120{\textsuperscript{***}} 
    \end{tabularx}%

  \end{block}%
  
\end{frame}

\begin{frame}
\frametitle{H.2.2. Relative resources more than absolute in gendered housework tasks?}
  \includegraphics[width=\textwidth]{figure9.jpeg}
\end{frame}

\begin{frame}
\frametitle{H.2.2. (US)}
  \includegraphics[width=\textwidth]{fig1.png}
\end{frame}

\begin{frame}
\frametitle{Cleaning (Canada)}
  \includegraphics[width=\textwidth]{figure10.jpeg}
\end{frame}

\begin{frame}
\frametitle{Cleaning US)}
  \includegraphics[width=\textwidth]{fig2.png}
\end{frame}

\begin{frame}
\frametitle{Shopping Canada}
  \includegraphics[width=\textwidth]{figure11.jpeg}
\end{frame}

\begin{frame}
\frametitle{Shopping US}
  \includegraphics[width=\textwidth]{fig3.png}
\end{frame}

\begin{frame}
\frametitle{Cultural explanations?}

Small differences between cultural groups

  \begin{enumerate}
  \item  In cooking and cleaning tasks among Asian and French Canadians, the autonomy framework testing variable (personal income) can explain more of the gender gap compared to any other cultural groups where it can barely account for any gender gap.
  \item There are some similarities - evidence of intersectionality but also evidence of cultural differences in meaning of housework types. Eg. shopping tasks among South Asian Canadians  
  \end{enumerate}
\end{frame}

\begin{frame}
\frametitle{Results}

\tiny
  \begin{block}{\centering
  \tiny\begin{tabularx}{\dimexpr\textwidth-2\tabcolsep}{@{}X@{}X@{}X@{}}\textcolor{white}{H1 Differences by housework task?} & \textcolor{white}{H2 Not the same explanation of the gap by tasks} & \textcolor{white}{H3 Cultural differences?}\end{tabularx}}%
  
  \centering
  \rowcolors{1}{RowColorOdd}{RowColorEven}%
  \begin{tabularx}{\dimexpr\textwidth-2\tabcolsep}
  {@{}X@{}X@{}X@{}}%
  H1.1. Gap narrower in gender-neutral. ‘Doing gender’
 & H2.1 economic factors are in fact more likely to explain the gender-neutral tasks more. ‘Doing gender’
 & H3.1/2 Some cultural differences.
 
 \\%
       {}
 & H2.2. Power matters more in more gendered housework. Relative resources in Canada
Bargaining with time matters more in more gendered housework. Time Availability in the US
 & {}
    \end{tabularx}%

  \end{block}%
  
\end{frame}

\begin{frame}
\frametitle{In a few words,}

Segregation analysis: gender matters in what type of the tasks are assigned to women. Some evidence in support of the ‘doing gender’ framework. \\~\\

GAP analysis: it is power or relative resources that matter more in the explanation of the gender gap in indoor routine housework. In shopping tasks, women and men bargain with time. \\~\\

Cultural differences exist.

\end{frame}

\begin{frame}
\frametitle{Conclusions (on Part I of the presentation)}

The division of housework is: 
 \\

\begin{itemize}
\item task segregation persists $=$ specialization into spheres persists
\item Labor markets in two countries need to address different issues to achieve gender equality 
\begin{itemize}
\item Canada \rightarrow equalize the pay/relative power
\item US \rightarrow equalize time
\end{itemize}
\item Cultural expectations of housework to be associated with women need addressing
\end{itemize}

\end{frame}

\title[Housework Research] %optional
{Housework Research and Other Countries}
 
\subtitle{ISSP Countries}

\author[Kolpashnikova, Kamila] % (optional, for multiple authors)
{Kamila Kolpashnikova\inst{1}, Man Yee Kan \inst{2}, \& Tsui-o Tai \inst{3}}
 
\institute[NTPU] % (optional)
{
  \inst{1}%
  Visiting Scholar\\
  Department of Sociology\\
  National Taipei University  \\~\\
  \inst{2}%
  Department of Sociology\\
  University of Oxford  \\~\\
  \inst{3}%
  Department of Sociology\\
  National Taipei University 
}

\begin{frame}
\maketitle
\end{frame}

\begin{frame}
\frametitle{Research Question}

studies consistently show that particularly men with more egalitarian views on gender issues are more likely to take on more housework than more traditional men
 \\~\\

But is it really so?

\end{frame}

\begin{frame}
\frametitle{How gender attitudes are formed?}

Socialization:
\begin{itemize}
\item Peers (Cohort) (Brewster \& Padavic 2000)
\item Period (Brewster \& Padavic 2000, Carter \& Borch 2005, Ciabattari 2001)
\item Education (Bolzendahl \& Myers 2004, Brewster \& Padavic 2000, etc.)
\item Within new ‘roles’ such as being married (Berk, 1985; Gupta 1999; Moore\& Vanneman, 2003)

\end{itemize}

\end{frame}

\begin{frame}
\frametitle{Data}

Socialization:
\begin{itemize}
\item ISSP data, 2002 and 2012
\item Countries (25): Australia, Bulgaria, Chile, Czech Republic, Denmark, France, Germany, GB, Hungary, Israel, Japan, Latvia, Mexico, Norway, Philippines, Poland, Portugal, Russia, Slovakia, Slovenia, Spain, Sweden, Switzerland, US, Taiwan

\end{itemize}

\end{frame}

\begin{frame}
\frametitle{ISSP Countries: Women Gender Attitudes and Housework Participation}
  \includegraphics[width=\textwidth, height=.77\textheight]{women.png}
\end{frame}

\begin{frame}
\frametitle{ISSP Countries: Women Gender Attitudes and Housework Participation}
  \includegraphics[width=\textwidth, height=.77\textheight]{men.png}
\end{frame}

\begin{frame}
\frametitle{Results}

Socialization:
\begin{itemize}
\item Using 3sls
\item I better show the output in the paper
\item we find that egalitarian attitudes can only explain the participation of women in housework. Men, however, who are more egalitarian, eschew housework. 

\end{itemize}

\end{frame}

\title[Housework Research] %optional
{Routine Housework in Japan}
 
\subtitle{STULA}

\author[Kolpashnikova, Kamila] % (optional, for multiple authors)
{Kamila Kolpashnikova\inst{1} \& Man Yee Kan \inst{2}}
 
\institute[NTPU] % (optional)
{
  \inst{1}%
  Visiting Scholar\\
  Department of Sociology\\
  National Taipei University  \\~\\
  \inst{2}%
  Department of Sociology\\
  University of Oxford 
}

\begin{frame}
\maketitle
\end{frame}

\begin{frame}
\frametitle{Japan and US}

\tiny
  \begin{block}{\centering\tiny\begin{tabularx}{\dimexpr\textwidth-2\tabcolsep}{@{}X@{}p{0.30\textwidth}@{}X@{}X@{}X@{}X@{}X@{}}\textcolor{white}{Variables} & 
\textcolor{white}{Description} & 
\textcolor{white}{Mean 
(Women)}& 
\textcolor{white}{Robust SE}& 
\textcolor{white}{Mean 
(Men)}& 
\textcolor{white}{Robust SE}& 
\textcolor{white}{Diff. 
in Means}
\end{tabularx}}%
  \centering
    \rowcolors{1}{RowColorOdd}{RowColorEven}%
    \begin{tabularx}{\dimexpr\textwidth-2\tabcolsep}{@{}X@{}p{0.30\textwidth}@{}X@{}X@{}X@{}X@{}X@{}}%
     Japan
 & Routine indoor housework (min.)
 & 154.099
 & (0.717)
 & 16.825
 & (0.238)
 & 137.274{\textsuperscript{***}}
 \\%
     USA
 & Routine indoor housework (min.)
 & 105.880
 & (1.992)
 & 39.355
 & (1.278)
 & 66.525{\textsuperscript{***}}
  
    \end{tabularx}%

  \end{block}%
  
\end{frame}

\begin{frame}
\frametitle{Education and Housework Participation of Women in Japan}

\tiny
  \begin{block}{\centering\tiny\begin{tabularx}{\dimexpr\textwidth-2\tabcolsep}{@{}p{0.30\textwidth}@{}X@{}X@{}X@{}}{} & 
\textcolor{white}{Model 1} & 
\textcolor{white}{Model 2}& 
\textcolor{white}{Model 3}
\end{tabularx}}%
  \centering
    \rowcolors{1}{RowColorOdd}{RowColorEven}%
    \begin{tabularx}{\dimexpr\textwidth-2\tabcolsep}{@{}p{0.30\textwidth}@{}X@{}X@{}X@{}}%
     Education in Years
 & -1.907{\textsuperscript{***}}
 & 1.737{\textsuperscript{***}}
 & 0.243
 \\%
     Paid Work
 & -0.129{\textsuperscript{***}}
 & -0.276{\textsuperscript{***}}
 & -0.321{\textsuperscript{***}}
 \\%
 Income
 & -13.35{\textsuperscript{***}}
 & -2.784{\textsuperscript{***}}
 & 11.09{\textsuperscript{***}}
 \\%
 Weekday
 & 7.146{\textsuperscript{***}}
 & 30.32{\textsuperscript{***}}
 & 55.19{\textsuperscript{***}}
 \\%
 Age
 & 10.15{\textsuperscript{***}}
 & 9.338{\textsuperscript{***}}
 & 15.29{\textsuperscript{***}}
 \\%
 Age*Age
 & -0.0891{\textsuperscript{***}}
 & -0.0859{\textsuperscript{***}}
 & -0.147{\textsuperscript{***}}
 \\%
 Constant
 & -100.6{\textsuperscript{***}}
 & -37.87{\textsuperscript{**}}
 & -132.3{\textsuperscript{***}}
 \\%
 Observations
 & 57,638
 & 43,474
 & 40,774
 \\%
 R-squared
 & 0.310
 & 0.163
 & 0.219
  
    \end{tabularx}%

  \end{block}%
  
\end{frame}


\begin{frame}
\frametitle{US Education and Housework, Women in 2006}

\tiny
  \begin{block}{\centering\tiny\begin{tabularx}{\dimexpr\textwidth-2\tabcolsep}{@{}p{0.30\textwidth}@{}X@{}X@{}X@{}}{} & 
\textcolor{white}{Model 1} & 
\textcolor{white}{Model 2}& 
\textcolor{white}{Model 3}
\end{tabularx}}%
  \centering
    \rowcolors{1}{RowColorOdd}{RowColorEven}%
    \begin{tabularx}{\dimexpr\textwidth-2\tabcolsep}{@{}p{0.30\textwidth}@{}X@{}X@{}X@{}}%
     Education in Years
 & -3.201{\textsuperscript{***}}
 & -4.071{\textsuperscript{*}}
 & -3.324{\textsuperscript{***}}
 \\%
     Paid Work
 & -0.104{\textsuperscript{***}}
 & -0.209{\textsuperscript{***}}
 & -0.245{\textsuperscript{***}}
 \\%
 Income
 & 5.384
 & -15.06
 & -24.08{\textsuperscript{***}}
 \\%
 Weekday
 & -3.858
 & 36.99{\textsuperscript{***}}
 & 26.31{\textsuperscript{***}}
 \\%
 Age
 & 4.389{\textsuperscript{***}}
 & 3.544
 & 7.321{\textsuperscript{***}}
 \\%
 Age*Age
 & -0.0341{\textsuperscript{***}}
 & -0.0284
 & -0.0739{\textsuperscript{***}}
 \\%
 Constant
 & 14.42
 & 94.49{\textsuperscript{*}}
 & 58.97
 \\%
 Observations
 & 1,904
 & 1,060
 & 2,116
 \\%
 R-squared
 & 0.135
 & 0.168
 & 0.210
  
    \end{tabularx}%

  \end{block}%
  
\end{frame}

\begin{frame}
\frametitle{Quantile Regressions}

\begin{columns}
\begin{column}{0.5\textwidth}
	\begin{center}
   \includegraphics[width=\textwidth, height=.77\textheight]{Japan.png}
   \end{center}
\end{column}
\begin{column}{0.5\textwidth}
    \begin{center}
     \includegraphics[width=\textwidth, height=.77\textheight]{US.png}
     \end{center}
\end{column}
\end{columns}
 	
\end{frame}

\title[Housework Research] %optional
{Tempograms in Time Use Research}
 
\subtitle{Life in Kyrgyzstan Dataset}

\author[Kolpashnikova, Kamila] % (optional, for multiple authors)
{Kamila Kolpashnikova\inst{1} \& Man Yee Kan \inst{2}}
 
\institute[NTPU] % (optional)
{
  \inst{1}%
  Visiting Scholar\\
  Department of Sociology\\
  National Taipei University  \\~\\
  \inst{2}%
  Department of Sociology\\
  University of Oxford 
}

\begin{frame}
\maketitle
\end{frame}
                        
\begin{frame}
\frametitle{Tempogram for Women}

   \includegraphics[width=\textwidth, height=.77\textheight]{tempW.png}
 	
\end{frame}

\begin{frame}
\frametitle{Tempogram for Men}

   \includegraphics[width=\textwidth, height=.77\textheight]{tempM.png}
 	
\end{frame}

\end{document}
